\section{Instruction Set Description}

\subsection{Instruction Syntax}

The instruction mnemonics are composed of a main part that indicates the sequence
of operations, followed by an optional prefix that specifies the data type
if it is not an integer word. Immediate operands always appear last in
instruction syntax, except for memory instructions where the offset is located
on the left of the base register.  For instance:

{\small
\begin{lstlisting}
addd    $r1 = $r2, $r3      // Integer add of $r2 and $r3 into $r1.
addd    $r1 = $r2, 127      // Integer add of $r2 and 127 into $r1.
addds   $r1 = $r2, $r3      // Saturated add of $r2 and $r3 into $r1.
lbs     $r1 = 127[$r2]      // Load byte sign-extended from $r2+127.
sd      127[$r2] = $p4      // Store double word $p4 into $r2+127.
\end{lstlisting}
}

Most instruction operands are either register or immediate values. The \KalrayK
core also uses \emph{modifier} operands, which have an alphanumeric name and
start by a '.' character. These modifiers always follow the instruction mnemonic
in the instruction syntax:

{\small
\begin{lstlisting}
cb.dnez   $r1, label        // Branch to label if $r1 is not zero.
lwz.xs $r1 = $r2[$r3]       // Load word zero extend at $r3 + ($r2 * 4).
\end{lstlisting}
}

\subsection{Instruction Bundle Encoding} \label{sec:encoding}

\subsubsection{Structure of Instruction Bundles}

Instruction bundle encoding of the \KalrayK core follows a simple pattern. An
instruction bundle is composed of up to 8 syllables (32-bit words), whose bit 31
(denoted as \texttt{parallel} or \texttt{P}) is dedicated to encode the end of
the bundle (end of bundle when zero). All syllables also dedicate bits 29 and 30
to encode the \emph{steering} information: \begin{description} \item[0b00] BCU
instruction syllables or immediate extension syllable.  \item[0b01] LSU
instruction syllable.  \item[0b10] MAU or FPU instruction syllable.  \item[0b11]
ALU0 or ALU1 instruction syllable.  \end{description}

An instruction bundle is encoded with the instruction syllables first, followed
by the immediate extension syllables if any. The instruction
syllables in a bundle are ordered: \begin{itemize}
\item Instruction syllable that must issue on the BCU.
\item Instruction syllable that must issue on the TCA.
\item Instruction syllables that must issue on the ALU0, or both ALU0 and ALU1.
\item Instruction syllables that may issue on the ALU0, the ALU1, or the tiny ALUs.
\item Instruction syllable that must issue on the MAU or the FPU.
\item Instruction syllable that must issue on the LSU.
\end{itemize}

\subsubsection{Immediate Extensions in Instruction Bundles}

The immediate extension syllables must appear after all the instruction
syllables.  In the immediate extension syllables, the steering bits are 0b00,
while the \texttt{exnum} field in bits 27 and 28 encodes which execution unit
should make use of the immediate extension:
\begin{description}
\item[0b00] Immediate extension for the instruction syllable issued on the ALU0.
\item[0b01] Immediate extension for the instruction syllable issued on the ALU1.
\item[0b10] Immediate extension for the instruction syllable issued on the MAU or the FPU.
\item[0b11] Immediate extension for the instruction syllable issued on the LSU.
\end{description}

Excess immediate extension syllables for a particular instruction syllable are
ignored. Immediate extension syllables for instruction syllables without
immediate operand are also ignored. This feature can be exploited to pad bundles
for the alignment of subsequent instruction bundles, as long as the total number
of syllables does not exceed 8:
\begin{itemize}
\item If there is no ALU instruction, pad the end of bundle with the NOP
syllable (0x00000000).
\item If there is an ALU0 or double ALU instruction without immediate operand, pad the
end of bundle with the NOP syllable.
\item If there is an ALU0 or double ALU instruction with and extended immediate operand,
pad the end of bundle with the NOP syllable.
\item If there is an ALU0 or double ALU instruction with non-extended immediate
operand, pad the end of bundle with an immediate extension syllable that
sign-extends the original immediate operand.
\end{itemize} Indeed, the NOP syllable is interpreted as an ALU0 immediate extension.

\subsection{Instruction Execution}

The execution of instructions is described using a pseudo-C language. This
language retains the C syntax, however, it operates on bool(eans) and int(egers)
of unbounded size. The pseudo-C builtin functions are listed in
Table~\ref{table:pseudoc-builtins}. The core stateful resources are listed in
Table~\ref{table:pseudoc-symbols}, including the register files that behave like
arrays, and a memory interface MEM that implement the methods listed in
Table~\ref{table:memory-interface}. Finally, the core-specific functions
are listed in Table~\ref{table:pseudoc-externals}.

\begin{table}[ht]
\begin{center}
\begin{tabular}{|l|l|} \hline
Identifier & Description \\ \hline
\_ABS & Absolute value \\
\_BMM\_$n$ & Bit Matrix Multiply $n\times n$ \\
\_BMT\_$n$ & Bit Matrix Transpose $n\times n$ \\
\_CBS.$n$ & Count bit set on $n$ bits \\
\_CLS.$n$ & Count leading sign on $n$ bits \\
\_CLZ.$n$ & Count leading zero on $n$ bits \\
\_CTZ.$n$ & Count trailing zero on $n$ bits \\
\_MAX & Maximum \\
\_MIN & Minimum \\
\_ROL\_$n$ & Rotate Left modulo $n$ bits \\
\_ROR\_$n$ & Rotate Right modulo $n$ bits \\
\_SAT.$n$ & Saturation to $n$ bits \\
\_SWAP\_$n$ & Swap the pairs of $n$ bits slices \\
\_SX\_$n$ & Sign extension on $n$ bits \\
\_THROW & Throw hardware trap \\
\_ZX\_$n$ & Zero extension on $n$ bits \\
\hline \end{tabular}
\end{center}
\caption{Pseudo-C builtin functions.}
\label{table:pseudoc-builtins}
\end{table}

\begin{table}[ht]
\begin{center}
\begin{tabular}{|l|l|} \hline
Identifier & Description \\ \hline
.cas & Memory compare and swap \\
.dflush & Memory data cache flush \\
.dflushl & Memory data cache flush line \\
.dinval & Memory data cache invalidate \\
.dinvall & Memory data cache invalidate line \\
.dpurge & Memory data cache purge \\
.dpurgel & Memory data cache purge line \\
.dtouchl & Memory data cache touch line \\
.dzerol & Memory data cache zero line \\
.fence & Memory fence \\
.iinval & Memory instruction cache invalidate \\
.iinvall & Memory instruction cache invalidate line \\
.dtouchl & Memory instruction cache touch line \\
.load & Memory load \\
.loadd & Memory load double \\
.loadu & Memory load unsigned \\
.loadud & Memory load unsigned double \\
.store & Memory store \\
.storeu & Memory store uncached \\
.wflush & Memory flush write buffer \\
\hline \end{tabular}
\end{center}
\caption{Memory interface methods.}
\label{table:memory-interface}
\end{table}

\begin{table}[ht]
\begin{center}
\begin{tabular}{|l|l|} \hline
Identifier & Description \\ \hline
SFR[] & System Function Registers \\
GPR[] & General Purpose Registers \\
PGR[] & Paired General Registers \\
QGR[] & Quadrupled General Registers \\
XCR[] & Extension Compute Registers \\
XBR[] & Extension Block Registers \\
XVR[] & Extension Vector Registers \\
XTR[] & Extension Tile Registers \\
XMR[] & Extension Matrix Registers \\
MEM & Memory interface name \\
\hline \end{tabular}
\end{center}
\caption{Pseudo-C symbols.}
\label{table:pseudoc-symbols}
\end{table}

\begin{table}[ht]
\begin{center}
\begin{tabular}{|l|l|} \hline
Identifier & Description \\ \hline
bwlu\_$n$ & Bit-wise look-up on $n$ bits \\
comparison\_$n$ & Comparison on $n$ bits \\
condition\_$n$ & Condition test on $n$ bits \\
fabs\_$n$ & Floating point absolute value on $n$ bits \\
fadd\_$n$\_$n$ & Floating point add on $n$ bits \\
fdma\_$n$\_$m$ & Floating point multiply add on $n$ bits, result on $m$ bits \\
fdms\_$n$\_$m$ & Floating point multiply subtract on $n$ bits, result on $m$ bits \\
ffma\_$n$\_$m$ & Floating point fuse multiply add on $n$ bits, result on $m$ bits \\
ffman\_$n$\_$m$ & Floating point fuse multiply add negate on $n$ bits, result on $m$ bits \\
ffms\_$n$\_$m$ & Floating point fuse multiply subtract on $n$ bits, result on $m$ bits \\
ffmsn\_$n$\_$m$ & Floating point fuse multiply subtract negate on $n$ bits, result on $m$ bits \\
fixed\_$n$ & Floating point conversion to fixed point on $n$ bits \\
fixedu\_$n$ & Floating point conversion to unsigned fixed point on $n$ bits \\
float\_$n$ & Fixed point conversion to floating point on $n$ bits \\
floatu\_$n$ & Unsigned fixed point conversion to floating point on $n$ bits \\
floatcomp\_$n$ & Floating point compare on $n$ bits \\
fmax\_$n$ & Floating point maximum on $n$ bits \\
fmin\_$n$ & Floating point minimum on $n$ bits \\
fmul\_$n$\_$m$ & Floating point multiply on $n$ bits, result on $m$ bits \\
fmuln\_$n$\_$m$ & Floating point multiply negate on $n$ bits, result on $m$ bits \\
fnarrow\_$n$\_$m$ & Floating point narrow from $n$ bits to $m$ bits \\
fneg\_$n$ & Floating point negate on $n$ bits \\
fsbf\_$n$\_$m$ & Floating point subtract $n$ bits, result on $m$ bits \\
fsdiv\_$n$ & Floating point seed for division on $n$ bits \\
fsisr\_$n$ & Floating point seed for inverse square root on $n$ bits \\
fwiden\_32\_64 & Floating point widen single to double \\
\hline \end{tabular}
\end{center}
\caption{Pseudo-C external functions.}
\label{table:pseudoc-externals}
\end{table}

\clearpage

\newpage
%\ifx\flodraft\undefined
  \input{Instruction}
%\fi

\subsection{Synthetic Instructions}

Synthetic instructions are special cases of regular instructions, where one or
more operands are forced to constant values.

\newpage
\ifx\flodraft\undefined
  \input{Synthetic}
\fi


\subsection{Format Descriptions}

\ifx\flodraft\undefined
  \input{Format}
\fi


\subsection{Instruction Encoding Tables}

\paragraph{Note} Instructions whose name are suffixed with '*' are those which
are not yet described in this document or implemented in the software tools
(instruction set simulators, assembler). Moreover, we use the following
shortcuts in the table:

\begin{center}
\begin{tabular}{r@{\hskip 5mm$\leftrightarrow$\hskip 5mm}l}
  Abbreviation & Interpretation \\
  \hline
  b\_7\_5 & bits\_7\_5\\
  b\_9\_8 & bits\_9\_8\\
  comp    &    comparison\\
  cond & condition\\
  fcomp & floatcomp\\
  N    &    negate\\
  P    &    parallel\\
%  reg\{X,Y,\ldots\}    &    register\{X,Y,\ldots\}\\
  rnd & rounding\\
  scal & scaling\\
\end{tabular}
\end{center}

\begin{center}
  \input{Opcode}
\end{center}


\subsection{Instruction Reservation Tables}

\input{Resource}

The instruction scheduling constraints of the \KalrayK core with regards to
execution resources are exactly represented by \emph{reservation tables}. A
reservation table has one row per core execution resource, and one column
per execution cycle relative to the instruction issue date. Each entry in a
reservation table records the amount of execution resource required. Parallel
and pipelined instruction execution is allowed as long as the sum of required
execution resources in all the instruction reservation tables does not exceed
the execution resources available in the core, see
Table~\ref{table:resources}.

\input{Reservation}

\newpage
\subsection{Instruction Constraints} \label{sec:constraints}

The \KalrayK core bundling constraints are implied by the instruction
scheduling constraints on dependence latencies and execution resources. In
particular, a bundle is valid if the sum of the first column of each reservation
table of the bundled instructions does not exceed the availability of execution
resources given in Table~\ref{table:resources}.

\subsubsection{Instruction Bundling Constraints}

\paragraph{Rule 1}
Two data dependent instructions may not be scheduled within a single bundle.

\paragraph{Rule 2}
Two instructions of the same bundle must not target the same architectural
resource, otherwise the update value of this resource is unpredictable.

\paragraph{Rule 3}
(Specialization of Rule 2) An {ALU} instruction that produces a carry
must not be bundled with a \texttt{SET} or \texttt{WFXL} instruction on CS.

\paragraph{Rule 4}
(Specialization of Rule 2) An {ALU} or {MAU} instruction that produces an IEEE~754
floating point flag must not be bundled with a \texttt{SET} or \texttt{WFX*} instruction on
CS.

\paragraph{Rule 5}
If a \texttt{GET}, \texttt{IGET} or \texttt{WAITIT} instruction is issued within a bundle, then the BCU and the tiny
ALU unit inside the MAU are used.  As a consequence: \begin{itemize}
\item No {MAU} instruction is allowed in this bundle
\item Only 2 tiny {ALU} instructions may be used in this bundle (one on each ALU).
\end{itemize}
This rule is implicit and covered by the instruction reservation tables.

% FIXME Bostan legacy, should be reviewed
\subsubsection{Instruction Scheduling Constraints}

\paragraph{Rule 6 -- Mandatory}
PC must not be the target of a \texttt{hfx*} or a \texttt{set} instruction, else
a TRAP.OPCODE is generated.  A simple branch may be used to modify the PC if
needed.

\paragraph{Rule 7 -- Mandatory}
The following instructions may only be issued alone in the bundle: \begin{itemize}
\item \texttt{RFE}
\item \texttt{SCALL}
\item \texttt{AWAIT}
\item \texttt{SLEEP}
\item \texttt{STOP}
\item \texttt{BARRIER}
\item \texttt{LOOPDO}
\item \texttt{TLBREAD}
\item \texttt{TLBWRITE}
\item \texttt{TLBIINVAL}
\item \texttt{TLBDINVAL}
\item \texttt{TLBPROBE}
\item \texttt{SET/WFXL} on PS/MMC/*OW
\end{itemize}

\paragraph{Rule 8 -- Mandatory}
LS, LE and LC must not be modified within the hardware loop.

\paragraph{Rule 9 -- Mandatory}
The instructions that enable or disable hardware loops must not be used inside
the corresponding loop.

\paragraph{Rule 10 -- Mandatory}
To avoid any semantic ambiguity and simplify the design of the hardware, branching instructions, including those triggering implicit refetches,
are not allowed at a hardware loop last bundle (generates OPCODE traps). Those are: \begin{itemize}
\item \texttt{CALL}
\item \texttt{ICALL}
\item \texttt{CB}
\item \texttt{GOTO}
\item \texttt{IGOTO}
\item \texttt{RET}
\item \texttt{RFE}
\item \texttt{SCALL}
\item \texttt{BARRIER}
\item \texttt{LOOPDO}
\item \texttt{TLBREAD}
\item \texttt{TLBWRITE}
\item \texttt{TLBIINVAL}
\item \texttt{TLBDINVAL}
\item \texttt{TLBPROBE}
\item \texttt{SET/WFXL} on PS
\end{itemize}

\paragraph{Rule 11 -- Mandatory}
In order to ensure a correct behavior while executing an hardware loop, a
re-fetch must be generated after a set instruction involving LS, LE or LC
registers. This re-fetch can be done through a \texttt{BARRIER} instruction, but also
through the RFE or RET instructions.

\paragraph{Rule 12 -- Mandatory}
Similarly, a refetch has to be generated after an \texttt{IINVAL*} instruction. This
re-fetch can be done through a \texttt{BARRIER} instruction, but also through the \texttt{RFE} or
RET instructions.

\paragraph{Rule 13 -- Mandatory}
After changing the value of the rounding mode (RM) field of CS, a \texttt{BARRIER}
instruction must be issued before any MAU/FPU instruction, to guarantee that
these instructions will see the new value of RM.

\paragraph{Rule 14 -- Mandatory}
Similarly, after changing the value of PMC, a \texttt{BARRIER} is needed before a \texttt{GET}
instruction on the concerned registers among PM0, PM1, PM2 and PM3, to make sure
that the correct value will be read.

\paragraph{Rule 15 -- Performance}
For hardware simplicity, there are no bypasses on system function registers.  Then, in case
of RAW dependency, the core will stall until the write
completion. However, two specific cases are optimized:
\begin{itemize}
\item A \texttt{RET} instruction after a RA update will be executed seeing immediately the
  new RA value, without stalls.
\item An instruction which involves the carry flag after a CS update
  will be executed seeing immediately the new carry value, without stalls.
\end{itemize}



